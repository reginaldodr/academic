\documentclass[a4paper,addpoints,12pt]{exam}

%------------ comandos para gerar a prova e o gabarito -----------------------
\usepackage{currfile}
\usepackage{xstring}
\IfSubStr*{\currfilename}{gabarito}{ \printanswers }{ \noprintanswers }
%-----------------------------------


\usepackage{pythontex} 
%\usepackage[]{qrcode}

\usepackage{enumitem}

%\usepackage{showframe}

\usepackage[utf8]{inputenc}
\usepackage[portuges]{babel}
\usepackage{graphicx}
\usepackage{wrapfig}
\usepackage{color}
\usepackage{multicol}

%-----------tikz libraries ------------
\usepackage{tikz}
\usetikzlibrary{arrows,positioning} %positioning to break text in node
%---------------------------------------

\usepackage{enumerate}
\usepackage{amsmath,amssymb}
\usetikzlibrary{patterns}
\usepackage{array,booktabs}
\usepackage{mathtools}
\usepackage{dashrule}
\usepackage{caption}

\SolutionEmphasis{\color{blue}}


\qformat{\textbf{Questão \thequestion.} \dotfill  \rule{1.5cm}{.5pt}/ {\color{blue}\totalpoints\ pts}}

\pointpoints{pt}{pts}
\bonuspointname{\ bonus}
%\unframedsolutions
%\SolutionEmphasis
\pointformat{{\color{blue} \thepoints}}


%--------------   Critérios de correçaõ ----------------------------------

\usetikzlibrary{shapes,snakes}
\usepackage{environ}




\tikzstyle{mybox} = [draw=black, very thick,
rectangle, rounded corners, inner sep=10pt, inner ysep=20pt]
\tikzstyle{fancytitle} =[fill=white, text=red]




\NewEnviron{blc}[1]{\begin{tikzpicture}
	\node [mybox] (box){%
		\begin{minipage}{0.95\textwidth}
		\BODY
		\end{minipage}
	};
	\node[fancytitle, right=10pt] at (box.north west) {#1};
	\end{tikzpicture}
}


%---------------------------------------------------------------------------




%----------------------  Caixa para Instruções  --------------------------------


\tikzstyle{mybox2} = [draw=black, very thick,
rectangle, rounded corners, inner sep=0pt, inner ysep=9pt]
\tikzstyle{fancytitle2} =[fill=white, text=black]


%---------------------------------------------------------------------------


%------------------------ footenote without mark-----------------------------

\newcommand\blfootnote[1]{%
	\begingroup
	\renewcommand\thefootnote{}\footnote{#1}%
	\addtocounter{footnote}{-1}%
	\endgroup
}


%--------------------------------------------------------------------------
\renewcommand{\solutiontitle}{\noindent\textbf{Solução:}\enspace}
\newcommand{\dps}{\displaystyle}
\newcommand{\R}{\mathbb{R}}
\newcommand{\sen}{\operatorname{sen}}
\renewcommand{\half}{,5}
\newcommand{\fm}{\textsuperscript{\b{a}\ }}
\newcommand{\mc}{\textsuperscript{\b{0}\ }}
\newcommand{\D}{\dps\frac{d}{dx}}
\newcommand{\tg}{\operatorname{tg}}
\newcommand{\cotg}{\operatorname{cotg}}
\newcommand{\cosec}{\operatorname{cosec}}
\newcommand{\arcsen}{\operatorname{arcsen}}
\newcommand{\arctg}{\operatorname{arctg}}
\newcommand{\arccosec}{\operatorname{arccosec}}
\newcommand{\arccotg}{\operatorname{arccotg}}
\newcommand{\arcsec}{\operatorname{arcsec}}
\newcommand{\im}{\operatorname{Im}}
\newcommand{\lap}{\mathcal{L}}
\newcommand{\senh}{\operatorname{senh}}
\newcommand{\pyl}[1]{\py{sp.latex(#1)}}

\setlength{\textwidth}{7.3in} \setlength{\oddsidemargin}{-.7in}

\setlength{\evensidemargin}{0in}

\setlength{\marginparwidth}{0.in} \setlength{\textheight}{9.8in}

\setlength{\topmargin}{-0.2in} \setlength{\footskip}{0.0in}

\setlength{\headsep}{0.1in}

\begin{document}
	

\header{\includegraphics[scale=0.08]{brasao-cor.png}}
{\sc \scriptsize Universidade Federal Fluminense \\
%Instituto de Humanidades e Saúde -- RHS\\
Departamento de Ciências da Natureza\\
Campus de Rio das Ostras \\
\textbf{Professor Reginaldo Demarque}\\}
{\scriptsize 
\textbf{{\color{red} 1\fm Prova de}}\\
\textbf{Cálculo III}\\
\textbf{\textcolor{red}{27/09/2023} -- 2023-2}\\
\textbf{\textcolor{blue}{Turma M1}}\\
}
	
	
	\ifprintanswers
	\begin{center}
		\noindent \\
	{\Large \textbf{\textcolor{red}{Gabarito}}}\\
	\end{center}
	\noindent \hrulefill  

	\else
\noindent
\begin{tikzpicture}
\node [mybox2] (box){%
	\begin{scriptsize}
	\begin{minipage}{0.51\textwidth}
	\begin{itemize}[noitemsep]
	\item Só poderá sair da sala quem entregar definitivamente a avaliação.
	\item Responda de forma clara e organizada.
	\item As soluções devem ser fundamentadas no conteúdo ensinado.
	\item A interpretação faz parte dos critérios de avaliação.
	\item Só é permitido o uso de calculadora científica. 
	\end{itemize}
	\end{minipage}
	\begin{minipage}{0.55\textwidth}
	\begin{itemize}[noitemsep]
\item Não é permitido o compartilhamento de material.
\item Uma questão com mais de uma solução será anulada.
\item Resposta final correta com solução incorreta será anulada.
\item Detectada fraude, mesmo posteriormente, a avaliação será anulada.
	
	\end{itemize}
	\end{minipage}
	\end{scriptsize}
};
\node[fancytitle2, right=15pt] at (box.north west) {{\color{red}Instruções}};
%\node[fancytitle2, rounded corners] at (box.east) {$\clubsuit$};
\end{tikzpicture}%
	
	
	\bigskip
	\noindent Nome:\hrulefill \ Nota:\ \rule{1.5cm}{.5pt} / {\color{blue} \numpoints\ pts}.
	\bigskip
\fi
	
	
\bracketedpoints


\begin{questions}

%------------------------------------------
%-------------  Questão 1 -----------------
%------------------------------------------

\begin{pycode}
import sympy as sp

t,x=sp.symbols('t x')

def pt(Y):
 return(sp.latex((Y[0],Y[1],Y[2])))
 

def pt2(Y):
  return(sp.latex((Y[0],Y[1])))


A=sp.Matrix([1,0,1])
B=sp.Matrix([3,1,4])

ra=A+t*(B-A)

c=sp.Matrix([1,-1])
r=4

rb=sp.Matrix([c[0]+r*sp.cos(t),c[1]+r*sp.sin(t)])

y=sp.sqrt(x)

rc=sp.Matrix([x,y])
 


\end{pycode}

\question Dê uma parametrização das seguintes curvas:



\begin{parts}
\part[1] O segmento de reta ligando os pontos $A=(1,0,1)$ e $B=(3,1,4)$.


\part[1]  O círculo de centro $(1,-1)$ e raio $r=4$.

\part[1] O gráfico da função $y=\sqrt{x}$.

\end{parts}

\begin{solution}
\begin{parts}
\part  Note que $\overrightarrow{AB} =\py{pt(B-A)}$, daí,
\[\vec{r}(t)=\py{pt(A)}+t\py{pt(B-A)}=\py{pt(ra)},\ t\in [0,1].\]

\part 
\[\vec{r}(t)=\py{pt2(rb)},\, t\in[0,2\pi].\]

\part 
\[\vec{r}(x)=\py{pt2(rc)},\ x\in [0,+\infty[ .\]
\end{parts}
\end{solution}


\newpage

%------------------------------------------
%-------------  Questão 2 -----------------
%------------------------------------------

\begin{pycode}
import sympy as sp

t=sp.symbols('t',positive=True)

a=0
b=sp.pi

def pt(Y):
 return(sp.latex((Y[0],Y[1],Y[2])))
 
def r(t):
	return sp.Matrix([t**2, sp.cos(t)+t*sp.sin(t),sp.sin(t)-t*sp.cos(t)])

dr=sp.simplify(r(t).diff(t))
ndr=dr.norm()

I=sp.integrate(sp.simplify(ndr),t)
Id=sp.simplify(sp.integrate(sp.simplify(ndr),(t,a,b)))




\end{pycode}

\question[2] Calcule o comprimento da curva $\vec{r}(t)=\py{pt(r(t))}$, $\pyl{a}\leq t\leq \pyl{b}$.



\begin{solution} 
 Note que 
\[\vec{r}\,^\prime(t)=\py{pt(dr)}\Rightarrow \|\vec{r}\,^\prime(t)\|=\pyl{ndr}=\pyl{sp.simplify(ndr)}.\]
Com isso, temos que
\[L=\int_{\pyl{a}}^{\pyl{b}} \|\vec{r}\,^\prime(t)\|\,dt= \int_{\pyl{a}}^{\pyl{b}}\pyl{sp.simplify(ndr)}\,dt=\pyl{I}\Big\vert_{t=\pyl{a}}^{t=\pyl{b}} =\pyl{Id}.\]


\end{solution}


\newpage
%------------------------------------------
%-------------  Questão 3 -----------------
%------------------------------------------
\begin{pycode}
import sympy as sp

t=sp.symbols('t',positive=True)


def pt(Y):
 return(sp.latex((Y[0],Y[1],Y[2])))
 
def r(t):
	return sp.Matrix([t,t,t**2/2])

dr=sp.simplify(r(t).diff(t))
d2r=sp.simplify(r(t).diff(t,2))

ndr=dr.norm()

A=dr.cross(d2r)
n1=A.norm()

dv=ndr.diff(t)

k=n1/ndr**3

T=dr/ndr

dT=T.diff(t)

at=dv
an=ndr**2*k


\end{pycode}

\question[3] Uma partícula se move de acordo com a parametrização $\vec{r}(t)=\py{pt(r(t))}$, $t\in \R$.
\begin{parts}
\part Determine a curvatura da curva.
\part Determine as componentes tangente e normal da aceleração. 
\end{parts}

\begin{solution}
\noindent 

\begin{parts}
\part \noindent
\begin{equation*}
\begin{split}
&\vec{v}(t)=\vec{r}\,^\prime(t)=\py{pt(dr)}\Rightarrow v(t)= \|\vec{r}\,^\prime(t)\|=\pyl{ndr}.\\
&\vec{a}(t)=\vec{r}\,^{\prime\prime}(t)=\py{pt(d2r)}\Rightarrow \vec{v}\times \vec{a}=\py{pt(A)}\\
&\kappa(t)=\frac{\|\vec{v}\times\vec{a}\|}{v^3}=\pyl{k}
\end{split}
\end{equation*}



\part Como
\[v'(t)=\pyl{dv},\]
temos que 
\[\vec{a}(t)=v'(t)\vec{T}+v^2(t)\kappa(t)\vec{N}=\pyl{at}\vec{T}+\pyl{an}\vec{N}.\]
\end{parts}


\end{solution}


\newpage
%------------------------------------------
%-------------  Questão 4 -----------------
%------------------------------------------

\begin{pycode}
import sympy as sp

t,x,y,z=sp.symbols('t x y z',positive=True)


def pt(Y):
 return(sp.latex((Y[0],Y[1],Y[2])))
 
def r(t):
	return sp.Matrix([2*sp.cos(t),2*sp.sin(t),sp.exp(t)])

dr=sp.simplify(r(t).diff(t))

vn=sp.Matrix([sp.sqrt(3),1,0])
X=sp.Matrix([x,y,z])
eq=sp.Eq(vn.dot(X),1)

eq2=sp.Eq(dr.dot(vn),0)
sol=sp.solve(eq2,t)
t0=sol[0]

\end{pycode}

\question[2] Encontre o ponto da curva $\vec{r}(t)=\py{pt(r(t))}$, $0\leq t\leq \pi$, em que a reta tangente é paralela ao plano $\pyl{eq}$.

\begin{solution}
Basta ver em quais pontos $\vec{r}\,^\prime(t)=\py{pt(dr)}$ é ortogonal a \\ $\vec{n}=\py{pt(vn)}$, o vetor normal do plano. Com efeito,
\[\vec{r}\,^\prime(t)\cdot \vec{n}=0\Rightarrow\pyl{eq2}\Rightarrow \tan(t)=\frac{1}{\sqrt{3}},\]
donde concluímos que $t=\pyl{t0}$. Portanto o ponto da curva buscado é:
\[\vec{r}\left(\pyl{t0}\right)=\py{pt(r(t0))}\]

\end{solution}

\end{questions}



\end{document}





%\include{derivadas}


