\documentclass[a4paper,addpoints,12pt]{exam}

%------------ comandos para gerar a prova e o gabarito -----------------------
\usepackage{currfile}
\usepackage{xstring}
\IfSubStr*{\currfilename}{gabarito}{ \printanswers }{ \noprintanswers }
%-----------------------------------


\usepackage{pythontex} 
%\usepackage[]{qrcode}

\usepackage{enumitem}

%\usepackage{showframe}

\usepackage[utf8]{inputenc}
\usepackage[portuges]{babel}
\usepackage{graphicx}
\usepackage{wrapfig}
\usepackage{color}
\usepackage{multicol}

%-----------tikz libraries ------------
\usepackage{tikz}
\usetikzlibrary{arrows,positioning} %positioning to break text in node
%---------------------------------------

\usepackage{enumerate}
\usepackage{amsmath,amssymb}
\usetikzlibrary{patterns}
\usepackage{array,booktabs}
\usepackage{mathtools}
\usepackage{dashrule}
\usepackage{caption}

\SolutionEmphasis{\color{blue}}


\qformat{\textbf{Questão \thequestion.} \dotfill  \rule{1.5cm}{.5pt}/ {\color{blue}\totalpoints\ pts}}

\pointpoints{pt}{pts}
\bonuspointname{\ bonus}
%\unframedsolutions
%\SolutionEmphasis
\pointformat{{\color{blue} \thepoints}}


%--------------   Critérios de correçaõ ----------------------------------

\usetikzlibrary{shapes,snakes}
\usepackage{environ}




\tikzstyle{mybox} = [draw=black, very thick,
rectangle, rounded corners, inner sep=10pt, inner ysep=20pt]
\tikzstyle{fancytitle} =[fill=white, text=red]




\NewEnviron{blc}[1]{\begin{tikzpicture}
	\node [mybox] (box){%
		\begin{minipage}{0.95\textwidth}
		\BODY
		\end{minipage}
	};
	\node[fancytitle, right=10pt] at (box.north west) {#1};
	\end{tikzpicture}
}


%---------------------------------------------------------------------------




%----------------------  Caixa para Instruções  --------------------------------


\tikzstyle{mybox2} = [draw=black, very thick,
rectangle, rounded corners, inner sep=0pt, inner ysep=9pt]
\tikzstyle{fancytitle2} =[fill=white, text=black]


%---------------------------------------------------------------------------


%------------------------ footenote without mark-----------------------------

\newcommand\blfootnote[1]{%
	\begingroup
	\renewcommand\thefootnote{}\footnote{#1}%
	\addtocounter{footnote}{-1}%
	\endgroup
}


%--------------------------------------------------------------------------
\renewcommand{\solutiontitle}{\noindent\textbf{Solução:}\enspace}
\newcommand{\dps}{\displaystyle}
\newcommand{\R}{\mathbb{R}}
\newcommand{\sen}{\operatorname{sen}}
\renewcommand{\half}{,5}
\newcommand{\fm}{\textsuperscript{\b{a}\ }}
\newcommand{\mc}{\textsuperscript{\b{0}\ }}
\newcommand{\D}{\dps\frac{d}{dx}}
\newcommand{\tg}{\operatorname{tg}}
\newcommand{\cotg}{\operatorname{cotg}}
\newcommand{\cosec}{\operatorname{cosec}}
\newcommand{\arcsen}{\operatorname{arcsen}}
\newcommand{\arctg}{\operatorname{arctg}}
\newcommand{\arccosec}{\operatorname{arccosec}}
\newcommand{\arccotg}{\operatorname{arccotg}}
\newcommand{\arcsec}{\operatorname{arcsec}}
\newcommand{\im}{\operatorname{Im}}
\newcommand{\lap}{\mathcal{L}}
\newcommand{\senh}{\operatorname{senh}}
\newcommand{\pyl}[1]{\py{sp.latex(#1)}}

\setlength{\textwidth}{7.3in} \setlength{\oddsidemargin}{-.7in}

\setlength{\evensidemargin}{0in}

\setlength{\marginparwidth}{0.in} \setlength{\textheight}{9.8in}

\setlength{\topmargin}{-0.2in} \setlength{\footskip}{0.0in}

\setlength{\headsep}{0.1in}

\begin{document}
	

\header{\includegraphics[scale=0.08]{brasao-cor.png}}
{\sc \scriptsize Universidade Federal Fluminense \\
%Instituto de Humanidades e Saúde -- RHS\\
Departamento de Ciências da Natureza\\
Campus de Rio das Ostras \\
\textbf{Professor Reginaldo Demarque}\\}
{\scriptsize 
\textbf{{\color{red} 1\fm Prova de}}\\
\textbf{Cálculo II}\\
\textbf{\textcolor{red}{25/10/2023} -- 2023-2}\\
\textbf{\textcolor{blue}{Turma C1}}\\
}
	
	
	\ifprintanswers
	\begin{center}
		\noindent \\
	{\Large \textbf{\textcolor{red}{Gabarito}}}\\
	\end{center}
	\noindent \hrulefill  

	\else
\noindent
\begin{tikzpicture}
\node [mybox2] (box){%
	\begin{scriptsize}
	\begin{minipage}{0.51\textwidth}
	\begin{itemize}[noitemsep]
	\item Só poderá sair da sala quem entregar definitivamente a avaliação.
	\item Responda de forma clara e organizada.
	\item As soluções devem ser fundamentadas no conteúdo ensinado.
	\item A interpretação faz parte dos critérios de avaliação.
	\item Só é permitido o uso de calculadora científica. 
	\end{itemize}
	\end{minipage}
	\begin{minipage}{0.55\textwidth}
	\begin{itemize}[noitemsep]
\item Não é permitido o compartilhamento de material.
\item Uma questão com mais de uma solução será anulada.
\item Resposta final correta com solução incorreta será anulada.
\item Detectada fraude, mesmo posteriormente, a avaliação será anulada.
	
	\end{itemize}
	\end{minipage}
	\end{scriptsize}
};
\node[fancytitle2, right=15pt] at (box.north west) {{\color{red}Instruções}};
%\node[fancytitle2, rounded corners] at (box.east) {$\clubsuit$};
\end{tikzpicture}%
	
	
	\bigskip
	\noindent Nome:\hrulefill \ Nota:\ \rule{1.5cm}{.5pt} / {\color{blue} \numpoints\ pts}.
	\bigskip
\fi
	
	
\bracketedpoints


\begin{questions}

%------------------------------------------
%-------------  Questão  -----------------
%------------------------------------------

\begin{pycode}
import sympy as sp

x=sp.symbols('x')


fa=x**3+x**4
Ia=sp.integrate(fa,x)
Iad=sp.integrate(fa,(x,1,2))

fb=1/(x*sp.sqrt(x**2+2))
Ib=sp.integrate(fb,x)

fc=x**4*sp.log(x)
Ic=sp.integrate(fc,x)

\end{pycode}

\question Calcule as integrais

\begin{multicols}{2}
\begin{parts}
\part[1] $\dps\int_1^2 \pyl{fa}\,dx$

\part[1] $\dps\int \pyl{fb}\, dx $

\part[1] $\dps\int\pyl{fc}\, dx $

\part[1] $\dps\int_{0}^\infty \frac{e^x}{e^{x}+3}\, dx $
\end{parts}

\end{multicols}

\begin{solution}
\begin{parts}
	\part 
\[\int_1^2 \pyl{fa}\,dx= \pyl{Ia}\Big\vert_{x=1}^{x=2}=\pyl{Iad}. \]

	\part Vamos usar a substituição trigonométrica. Primeiramente, note que
\[\sqrt{x^2+2}=\sqrt{2\left(1+\frac{x^2}{2}\right)}
=\sqrt{2}\sqrt{1+\left(\frac{x}{\sqrt{2}}\right)^2}.\] 
Com isso, vamos usar a seguinte substituição trigonométrica
\begin{equation*}
\begin{cases*}
x=\sqrt{2}\tan(\theta),\ -\frac{\pi}{2}< \theta < \frac{\pi}{2}\\
dx= \sqrt{2}\sec^2(\theta)\,d\theta.
\end{cases*}
\end{equation*}
Daí,
\begin{equation*}
\begin{split}
\int \pyl{fb}\,dx& = \frac{1}{\sqrt{2}}\int \frac{1}{x\sqrt{1+\left(\frac{x}{\sqrt{2}}\right)^2}}\,dx\\ 
&=\frac{1}{\sqrt{2}}\int \frac{1}{\sqrt{2} \tan(\theta)\sqrt{1+\tan^2(\theta)}}\sqrt{2}\sec^2(\theta)\,d\theta\\
&= \frac{1}{\sqrt{2}}\int \frac{1}{\tan(\theta)|\sec(\theta)|}\sec^2(\theta)\,d\theta\\
&= \frac{1}{\sqrt{2}}\int \frac{\sec(\theta)}{\tan(\theta)}\,d\theta= \frac{1}{\sqrt{2}}\int \frac{1}{\cos(\theta)}\frac{\cos(\theta)}{\sin(\theta)}\,d\theta\\
&= \frac{1}{\sqrt{2}}\int \frac{1}{\sin(\theta)}\,d\theta= \frac{1}{\sqrt{2}}\int \cosec(\theta)\,d\theta\\
&= \frac{1}{\sqrt{2}}\log|\cosec(\theta)-\cotg(\theta)|+C\\
&\stackrel{(\ast)}{=} \frac{1}{\sqrt{2}}\log\left|\frac{\sqrt{x^2+2}-\sqrt{2}}{x}\right|+C\\
\end{split}
\end{equation*}
\newpage

\begin{minipage}{0.6\textwidth}
$(\ast)$ Vamos voltar para a variável anterior. Usando-se as relações trigonométricas do triângulo retângulo ao lado vemos que
\[\cotg(\theta)=\frac{\sqrt{2}}{x} \ \text{ e } \cosec(\theta)=\frac{\sqrt{x^2+2}}{x}\]
\end{minipage}
\begin{minipage}{0.3\textwidth}
\begin{tikzpicture}
	% Define the coordinates
	\coordinate (A) at (0,0);
	\coordinate (B) at (4,0);
	\coordinate (C) at (0,2);
	
	% Draw the triangle
	\draw (A) -- node[below] {$\sqrt{2}$} (B) -- node[above right] {$\sqrt{x^2+2}$} (C) -- node[left] {$x$}  cycle;
	
	% Mark the right angle
	\draw (3,0.5) arc (150:180:1);
	
	% Label the angles
	\node at (2.5,.4) {$\theta$};
\end{tikzpicture}
\end{minipage}

\part Vamos resolver esta integral por partes. Fazendo 
\[
\begin{array}{ll}
	u=\log(x) & du=\frac{1}{x}dx\\
	\\
	dv=x^4dx & v=\frac{x^5}{5}, 
\end{array}
\]
temos
\[\int \pyl{fc}\,dx= \frac{x^5}{5}\log(x)-\int \frac{x^4}{5}\,dx=\pyl{Ic}+C.\]

\part Primeiramente vamos resolver a integral indefinida. Fazendo a substituição $u=e^x+3$, $du=e^x\,dx$ temos que
\begin{equation*}
\int \frac{e^x}{e^x+3}\,dx= \int\frac{du}{u}=\log|u|+C=\log(e^x+3)+c
\end{equation*}
Agora, vamos resolver a integral imprópria:
\begin{equation*}
\begin{split}
\int_0^\infty\frac{e^x}{e^x+3}\,dx& = \lim\limits_{b\to \infty} \int_0^b\frac{e^x}{e^x+3}\,dx=\lim\limits_{b\to\infty}\log(e^x+3)\Big\vert_{x=0}^{x=b}\\
& =\lim\limits_{b\to\infty}\left(\log(e^b+3)-\log(4)\right)=\infty.
\end{split}
\end{equation*}
\end{parts}
\end{solution}

\newpage

%------------------------------------------
%-------------  Questão  -----------------
%------------------------------------------

\question[2] Deterine o comprimento de arco do gráfico da função $y=\log(\cos(x))$ entre os pontos 
$x=0$ e $x=\pi/3$.


\begin{solution} 
Note que
\[y'=-\frac{\sin(x)}{\cos(x)}=-\tan(x).\]
Com isso, o comprimento de arco é dado por:
\begin{equation*}
\begin{split}
\int_0^{\pi/3}\sqrt{1+\tan^2(x)}\, dx& =\int_0^{\pi/3} \sqrt{\sec^2(x)}\,dx \\
&=\int_0^{\pi/3}\sec(x)\,dx=\log|\sec(x)+\tan(x)|\Big\vert_{x=0}^{x=\pi/3}=\log\left(2+\sqrt{3}\right).
\end{split}
\end{equation*}


\end{solution}

\newpage
%------------------------------------------
%-------------  Questão  -----------------
%------------------------------------------

\begin{pycode}
import sympy as sp

x=sp.symbols('x')

f=6-x**2

k=sp.solve(sp.Eq(f,2),x)

Id=sp.integrate(f**2-4,x)
I=2*sp.pi*sp.integrate(f**2-4,(x,0,k[1]))
	
\end{pycode}

\question[2] Considere o sólido obtido pela rotação, em torno do eixo $x$, da região limitada pelos gráficos das funções $y=\pyl{f}$ e $y=2$. 
\begin{parts}
\part Faça um esboço da região e do sólido.
\part Calcule o volume do sólido.
\end{parts}


\begin{solution}
Determinando os pontos de interseção
\[\pyl{f}=2\Rightarrow x^2=4 \Rightarrow  x=\pyl{k[0]} \text{ ou } x=\pyl{k[1]}.\]
Pelo gráfico podemos ver que o sólido é simétrico em relação ao eixo $y$. Usando o método dos discos, temos que o volume é dado por
\begin{equation*}
\begin{split}
V& =2\pi\int_0^2\pyl{f**2-4}\,dx=2\pi\int_0^2 \pyl{sp.expand(f**2-4)}\,dx\\
& =2\pi\left(\pyl{Id}\right)\Big\vert_{x=0}^{x=2}=\pyl{I}.
\end{split}
\end{equation*}	
\end{solution}


\end{questions}



\end{document}





%\include{derivadas}


